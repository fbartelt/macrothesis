\documentclass[dissertation,master,firstlang=english,secondlang=brazil]{macrothesis}
% The package by default will load the glossaries and natbib packages
% If you want to disable them, use option `noglossaries` or `nonatbib`
% FOR FINAL VERSION, USE THE OPTION 'final'
\usepackage{blindtext} % For dummy text
\usepackage{amsmath,amsfonts,amssymb}
\usepackage{verbatim}
\usepackage{textcomp}
%%% Thesis metadata %%%
\title{A Minimal Thesis Example}
\author{John Doe}
\advisorname{Dr. Jane Smith}
\coadvisorname{Dr. Foo Bar} % If no co-advisor, just comment this line
% To add a second co-advisor, use the command as following:
% \coadvisorname[Second Co-advisor]{First Co-advisor }

% The following commands set the university, research group, program, and location names. The default values are exactly:
% \universityname{Universidade Federal de Minas Gerais}
% \researchgroupname{MACRO Research Group - Mechatronics, Control, and Robotics}
% \programname{Graduate Program in Electrical Engineering}
% \locationname{Belo Horizonte, Brazil}

%%% Preambles %%%
% Dedication and Acknowledgments
\dedication{To my family.}
\epigraph{``To be or not to be, that is the question.''\\\hfill William Shakespeare}
\acknowledgments{
    This is an acknowledgments. If you want to change the title, you should use the command as following:

    ``\textbackslash acknowledgments[TITLE]\{Text\}''
}
\acknowledgmentssecond{
    Esse é um exemplo de agradecimento. Caso deseje alterar o título, você deve utilizar o comando da seguinte forma: ``\textbackslash acknowledgmentssecond[TITULO]\{Text\}''.
}
% Abstracts (in both languages)
\abstract{
    This is the abstract in the first language. If you want to change the title of the abstract, you should use the command as following: ``\textbackslash abstract[TITLE]\{Text\}''. This work is very cool.
}
\keywords{example; thesis; Macro; LaTeX; UFMG}
\abstractsecond{
    Esse é um exemplo de abstract. Caso deseje alterar o título do abstract, você deve utilizar o comando da seguinte forma: ``\textbackslash abstractsecond[TITULO]\{Text\}''.
    Esse trabalho é muito maneiro.
}
\keywordssecond{exemplo; tese; Macro; LaTeX; UFMG}

%%% Acronyms and Notations %%%

% The package by default will load the glossaries package
% If you want to disable it, use option `noglossaries`
% To add new acronyms, use the command as following:
\newacronym{uav}{UAV}{Uncrewed aerial vehicle}
\newacronym{dof}{DOF}{Degrees of freedom}
\newacronym{ufmg}{UFMG}{Universidade Federal de Minas Gerais}
% To add new notations, use the command as following:
\newnotation{Rn}{$\mathbb{R}^n$}{Euclidean space of dimension $n$}
\newnotation{frobnorm}{$\|\cdot\|_F$}{Frobenius norm}
\newnotation{N}{$\mathbb{N}$}{Set of natural numbers starting from zero or one depending on context}
\newnotation{Z}{$\mathbb{Z}$}{Set of all integers, both positive and negative including zero}
\newnotation{C}{$\mathbb{C}$}{Set of complex numbers, represented as $a + bi$ where $a, b \in \mathbb{R}$}
% You do not need to use the \gls{label} command, as the package will 
% automatically add the acronyms and notations to the list, whether they are
% used in the text or not.
% NOTE: If you use glossaries, every new entry must be defined before the
% \begin{document} command, i.e., here in the preamble.

% If you disable glossaries, the package will assume that there is a file
% Acronyms.tex and Notations.tex in the same folder as the main file
% We assume by default the titles Acronyms and Notations

% Adds the defense certificate and the cataloging page
% If no path is provided, raises a warning.
\fichacatalografica{template/fichacatalografica.pdf}
% \atadefesa{template/fichacatalografica.pdf}


\begin{document}
% Generate cover page
\maketitlepage
% Generate preamble page (dedication, acknowledgments, abstracts, 
% list of figures, list of tables, acronyms, and notations)
\preamblepage

%----------------------------------------------------------------------------------------
%	THESIS CONTENT - CHAPTERS
%----------------------------------------------------------------------------------------
\chapter{Introduction}
Para utilizar o pacote, simplesmente adicione o comando \texttt{\textbackslash usepackage\{macrothesis\}} ao preâmbulo do seu documento. Não delete o arquivo \texttt{template/logo.pdf}, já que ele é utilizado para gerar a capa do documento. 

Você está com a versão do pacote: \macrothesisversion. Verifique o repositório do pacote para ver as últimas atualizações: \url{https://github.com/fbartelt/macrothesis}. Caso a Release seja mais nova que a versão atual, baixe o pacote novamente. Se você estiver utilizando um template do Overleaf, muito provavelmente ele está desatualizado.
\section{Opções do pacote}
O pacote \texttt{macrothesis} possui algumas opções que podem ser utilizadas para personalizar o documento. As opções disponíveis são:
\begin{itemize}
    \item \texttt{final}: Muda o modelo para um formato mais parecido com o aceito pela UFMG (note que isso ainda continuará infringindo algumas normas, porém foi aceito assim pela biblioteca em 2025). Para mais detalhes, veja a \cref{sec:modeloufmg}.
    \item \texttt{raffo}: Utiliza ``Advisors:'' em vez de ``Advisor:'' e ``Co-advisor:'' (com as respectivas traduções). Ambos os nomes são alinhados à esquerda. Esta opção é essencial caso você seja orientando do Prof. Raffo. 
    \item \texttt{noglossaries}: Desabilita o pacote \texttt{glossaries} que é utilizado para gerenciar acrônimos e notações. Esta opção é útil caso você deseje utilizar um pacote diferente para gerenciar acrônimos e notações ou caso você deseje utilizar outras opções para o pacote. É possível definir uma página de acrônimos e notações manualmente utilizando os comandos \texttt{\textbackslash setnotationpage}.
    \item \texttt{nonatbib}: Desabilita o pacote \texttt{natbib} que é utilizado para gerenciar referências bibliográficas. Esta opção é útil caso você deseje utilizar um pacote diferente para gerenciar referências bibliográficas ou caso você deseje utilizar outras opções para o pacote.
    \item \texttt{nohyperref}: Desabilita o pacote \texttt{hyperref} que é utilizado para gerar links no documento. Esta opção é útil caso você deseje utilizar um pacote diferente para gerar links ou caso você deseje utilizar outras opções para o pacote.
    \item \texttt{dissertation}: Usa o termo ``Dissertation'' ao invés do padrão ``Thesis''.
    \item \texttt{thesis}: Usa o termo ``Thesis''. Não é necessário utilizar essa opção, pois é o padrão.
    \item \texttt{doctor}: Usa o título ``Doctor'' ao invés do padrão ``Master''.
    \item \texttt{master}: Usa o título ``Master''. Não é necessário utilizar essa opção, pois é o padrão.
    \item \texttt{firstlang=<lingua>}: Define a língua principal do documento. Por padrão, a língua principal é o inglês. As opções disponíveis são: \\\emph{\supportedlangs}. 
    \item \texttt{secondlang=<lingua>}: Define a língua secundária do documento. As opções disponíveis são as mesmas. Por padrão, a língua secundária é \texttt{none}, o que significa que não há uma língua secundária.
    \item \texttt{nolof}: Desabilita a lista de figuras.
    \item \texttt{notlot}: Desabilita a lista de tabelas.
\end{itemize}
Note que as opções de línguas somente automatizam os títulos como ``Abstract'', ``Resumo'', ``Contents'', etc, além das informações da folha de rosto. É possível escrever o texto em outras línguas, desde que esses títulos sejam alterados manualmente. Alguns ambientes já possuem opções para alterar o título manualmente.

As listas de figuras e tabelas são automaticamente desabilitadas caso não haja figuras ou tabelas no documento. Caso você queira desabilitar essas listas mesmo que haja figuras ou tabelas, você pode utilizar as opções \texttt{nolof} e \texttt{notlot}.

\subsection{Uso do pacote \texttt{glossaries}}
Caso deseje utilizar o pacote \texttt{glossaries}, você pode adicionar novos acrônimos e notações utilizando os comandos predefinidos \texttt{\textbackslash newacronym\{label\}\{sigla\}\{descrição\}} e \texttt{\textbackslash newnotation\{label\}\{notação\}\{descrição\}}, respectivamente. Por exemplo, o acrônimo \gls{uav} e a notação \gls{Rn} foram adicionados utilizando esses comandos. Os acrônimos serão ordenados em ordem alfabética e as notações em ordem de definição.

Não é necessário utilizar o comando \texttt{\textbackslash gls\{label\}} (ou algo similar) para referenciar os acrônimos e notações, pois o pacote \texttt{macrothesis} irá automaticamente adicionar todos os acrônimos e notações à lista, independentemente de serem utilizados no texto ou não. Apesar disso, você \textbf{deve} definir um label único para cada acrônimo e notação.

Esses termos devem ser adicionados no preâmbulo (antes do \verb|begin{document}|) da seguinte forma:
\begin{verbatim}
\newacronym{uav}{UAV}{Uncrewed aerial vehicle}
\newacronym{dof}{DOF}{Degrees of freedom}
\newacronym{ufmg}{UFMG}{Universidade Federal de Minas Gerais}

\newnotation{Rn}{$\mathbb{R}^n$}{Euclidean space of dimension $n$}
\newnotation{frobnorm}{$\|\cdot\|_F$}{Frobenius norm}

\begin{document}
...
\end{document}
\end{verbatim}
É possível adicionar as notações e acrônimos em arquivos separados, importando-se esses arquivos antes do preâmbulo utilizando \texttt{\textbackslash input\{acronyms\_notation.tex\}} por exemplo.

Note que o pacote \texttt{glossaries} será carregado por padrão com as opções \texttt{acronym} e \texttt{symbols}. Caso deseje modificar essas opções, você deve utilizar a opção \texttt{noglossaries} do pacote \texttt{macrothesis} e carregar o pacote \texttt{glossaries} manualmente.

\subsection{Uso do pacote \texttt{natbib}}
Caso deseje utilizar o pacote \texttt{natbib}, você pode utilizar os comandos \texttt{\textbackslash citep\{label\}} para citações indiretas
e \texttt{\textbackslash citet\{label\}} para citações diretas. Por exemplo, a referência \citep{Gallier2020} foi adicionada utilizando o comando \texttt{\textbackslash citep\{test\}} e a referência \citet{Lee2012} foi adicionada utilizando o comando \texttt{\textbackslash citet\{test\}}.

O pacote \texttt{natbib} é carregado por padrão com as opções \texttt{authoryear} e \texttt{round}. Caso deseje modificar essas opções, você deve utilizar a opção \texttt{nonatbib} do pacote \texttt{macrothesis} e carregar o pacote \texttt{natbib} manualmente.

\section{Uso do pacote \texttt{macrothesis}}
O pacote \texttt{macrothesis} é utilizado para gerar a capa, a folha de rosto, a dedicatória, os agradecimentos, os resumos, a lista de figuras, a lista de tabelas, a lista de acrônimos, a lista de notações, e a bibliografia. Para utilizar o pacote, basta adicionar os comandos \texttt{\textbackslash maketitlepage} e \texttt{\textbackslash preamblepage} ao início do seu documento (nessa ordem).

Para alterar o título, o autor, o orientador, etc. do documento, você deve utilizar os comandos \texttt{\textbackslash title\{Título\}}, \texttt{\textbackslash author\{Autor\}}, \texttt{\textbackslash advisorname\{Orientador\}}, etc. no preâmbulo. Por exemplo, para gerar a capa e a folha de rosto, você deve utilizar os comandos:
\begin{verbatim}
    \title{A Minimal Thesis Example}
    \author{John Doe}
    \advisorname{Dr. Jane Smith}
    \coadvisorname{Dr. Foo Bar}
    \universityname{Universidade Federal de Minas Gerais}
    \researchgroupname{MACRO Research Group - Mechatronics,
     Control, and Robotics}
    \programname{Graduate Program in Electrical Engineering}
    \locationname{Belo Horizonte, Brazil}
    ...
    \begin{document}
    \maketitlepage
    \preamblepage
    ...
    \end{document}
\end{verbatim}
O comando \texttt{\textbackslash coadvisorname\{Coorientador\}} é opcional e deve ser utilizado caso haja um coorientador. Para adicionar um segundo coorientador, você deve utilizar o comando da seguinte forma: \texttt{\textbackslash coadvisorname[Segundo Coorientador]\{Primeiro Coorientador\}}.

Para adicionar a dedicatória, epígrafe, os agradecimentos, e os resumos, você deve utilizar os comandos:
\begin{itemize}
    \item \texttt{\textbackslash dedication\{Dedicatória\}}
    \item \texttt{\textbackslash epigraph\{Epígrafe\}}
    \item \texttt{\textbackslash acknowledgments\{Agradecimentos\}}
    \item \texttt{\textbackslash acknowledgmentssecond\{Acknowledgments\}}
    \item \texttt{\textbackslash abstract\{Resumo\}}
    \item \texttt{\textbackslash abstractsecond\{Abstract\}}
\end{itemize}

Por exemplo, para adicionar a dedicatória e os agradecimentos, você deve utilizar os comandos:
\begin{verbatim}
    \dedication{To my family.}
    \epigraph{``To be or not to be, that is the question.''\\William Shakespeare}
    \acknowledgments{
        Long and emotional text here...
    }
    \acknowledgmentssecond{
        Tradução do Acknowledgments
    }
    \abstract{    
        Abstract in english...
    }
    \abstractsecond{
        Resumo em português...
    }
    ...
    \begin{document}
    \maketitlepage
    \preamblepage
    ...
    \end{document}
\end{verbatim}

Os agradecimentos, epígrafe e dedicatória são opcionais. Se não forem utilizados, o modelo irá ignorá-los.

Caso necessário, é possível modificar o título das seções de abstracts e acknowledgments utilizando os comandos da seguinte forma: \texttt{\textbackslash abstract[Título em sua língua]\{Texto\}} e \texttt{\textbackslash acknowledgments[Título em sua língua]\{Texto\}}.

Caso você não queira usar o pacote \texttt{glossaries}, a notação pode ser inserida através do comanndo \texttt{\textbackslash setnotationpage\{...\}}.

Para adicionar a ficha catalográfica, utilize o comando \texttt{\textbackslash fichacatalografica\{PATH\}}. O arquivo deve ser um PDF e o caminho deve ser relativo ao arquivo principal.

Para adicionar a ata de defesa (ou folha de aprovacao), utilize o comando \texttt{\textbackslash atadefesa\{PATH\}}. O arquivo deve ser um PDF e o caminho deve ser relativo ao arquivo principal.

Note que o modelo irá gerar um Warning caso não haja uma ficha catalográfica ou uma ata de defesa. Isso é feito para garantir que o usuário não esqueça de adicionar esses arquivos na versão final. Enquanto o trabalho estiver em desenvolvimento, você pode ignorar esse Warning. Como forma de exemplo, apenas foi adicionado um PDF de ficha catalográfica, note que há um Warning na compilação: \emph{Package macrothesis: Defense certificate (Ata de Defesa) not provided. This warning can be ignored in draft versions.}

Para um exemplo completo utilizando o modelo, veja o repositório \url{https://github.com/fbartelt/dissertation/blob/main/main.tex}

\subsection{Features}
O modelo macro também vem com algumas funcionalidades adicionais, como:
\begin{feature}
    \item Ambientes Theorem, Lemma, Proposition, Corollary, Definition, Remark, Example, Assumption, Claim, Conjecture, Fact, Problem, Solution, Criterion; \label{feat:envs}
    \item Listas de propriedades (\texttt{property}) e features (\texttt{feature}). \label{feat:lists}
\end{feature}
Essas listas e ambientes já vem com as definições necessárias pelo pacote \texttt{cref}. Por exemplo, os ambientes listados em \cref{feat:envs} podem ser feitos da seguinte forma:
\begin{definition}
    \label{def:example}
    This is an example of a definition. The label is automatically added to the list of definitions.
\end{definition}
\begin{definition}
    \label{def:example2}
    Um grupo é um conjunto não vazio $G$ com uma operação binária $\cdot$ que satisfaz as seguintes propriedades:
    \begin{property}
        \item Associatividade: Para todo $a,b,c \in G$, temos que $(a \cdot b) \cdot c = a \cdot (b \cdot c)$. \label{prop:associativity}
        \item Elemento neutro: Existe um elemento $e \in G$ tal que para todo $a \in G$, temos que $a \cdot e = e \cdot a = a$. \label{prop:identity}
        \item Inverso: Para todo $a \in G$, existe um elemento $b \in G$ tal que $a \cdot b = b \cdot a = e$. \label{prop:inverse}
        \item Fechamento: Para todo $a,b \in G$, temos que $a \cdot b \in G$. \label{prop:closure}
    \end{property}
\end{definition}

As \cref{def:example,def:example2} são exemplos de definição. A \cref{def:example2} é um exemplo de definição com uma lista de propriedades, assim pode-se referenciar que a \cref{prop:associativity} é uma propriedade de um grupo, assim como as \cref{prop:identity,prop:inverse,prop:closure}. Note que os plurais são controlados pelo pacote \texttt{cref}.

\begin{theorem}
    Um sistema modelado por Newton-Aclecio-Euler é muito mais eficiente do que um sistema modelado por Euler-Lagrange. \label{thm:example}
\end{theorem}
\begin{proof}
    A prova é trivial e deixada como exercício para o leitor.
\end{proof}

\begin{lemma}
    Um exemplo de lema. \label{lem:example}
\end{lemma}
\begin{proposition}
    Um exemplo de proposição. \label{propo:example}
\end{proposition}
\begin{corollary}
    Um exemplo de corolário. \label{cor:example}
\end{corollary}
\begin{remark}
    Um exemplo de observação. \label{rem:example}
\end{remark}
\begin{example}
    Um exemplo de exemplo. \label{ex:example}
\end{example}
\begin{assumption}
    Um exemplo de suposição. \label{ass:example}
\end{assumption}

\begin{conjecture}
    Um exemplo de conjectura. \label{cnj:example}
\end{conjecture}
\begin{claim}
    Um exemplo de afirmação. \label{clm:example}
\end{claim}
\begin{fact}
    Um exemplo de fato. \label{fact:example}
\end{fact}
\begin{problem}
    Um exemplo de problema. \label{prob:example}
\end{problem}
\begin{solution}
    Um exemplo de solução. \label{sol:example}
\end{solution}
\begin{criterion}
    Um exemplo de critério. \label{crit:example}
\end{criterion}

Note que os ambientes \cref{lem:example,propo:example,cor:example} necessitam de uma prova, enquanto os ambientes \cref{rem:example,ex:example,ass:example,cnj:example,clm:example,fact:example,prob:example,sol:example,crit:example} tem o símbolo $\qedsymbol$ adicionados automaticamente para tornar fácil a visualização do fim do ambiente.

O primeiro número desses ambientes se refere ao capítulo, enquanto o segundo número é compartilhado por todos eles, o que facilita encontrar essas referências por número.

\section{Modelo UFMG} \label{sec:modeloufmg}
A biblioteca da UFMG não aceita o modelo MACRO, que não está nas normas ABNT. Portanto, o ideal é enviar o modelo macro para o PPGEE e o modelo UFMG para a biblioteca. Para conferir as normas da UFMG acesse \url{https://repositorio.ufmg.br/static/politica/diretrizes-para-normalizacao-de-trabalhos-academicos-da-UFMG.pdf}.

\textbf{Utilize a opção \texttt{final} antes de enviar para a biblioteca}.

Dentre as reclamações que se teve até então (2025) a respeito do modelo MACRO:
\begin{itemize}
    \item Não é permitido numeração de folhas em romanos;
    \item Não deixe folhas em branco;
    \item A contagem de folhas não é a partir da capa, mas sim a partir da folha de rosto;
    \item A numeração só é visível a partir do primeiro capítulo (Introdução);
    \item O Resumo e Abstract devem ter apenas 1 página, com palavras-chave separadas por ponto e vírgula e espaçadas do resumo por 1 linha; Nenhum desses é indentado;
    \item Posicionar os títulos dos capítulos e dos elementos pré-textuais no canto esquerdo das páginas;
    \item Remover do sumário as listas, pois ele deve iniciar com o primeiro elemento textual (introdução);
    \item Corrigir a ordem sequencial do elementos pré-textuais. A ordem correta é:
    \begin{enumerate}
        \item Capa (obrigatória);
        \item Folha de rosto (obrigatória);
        \item Ficha catalográfica - (obrigatória);
        \item Folha de aprovação ou Ata de defesa - (obrigatória);
        \item Dedicatória(opcional);
        \item Agradecimento(opcional);
        \item Epígrafe(opcional);
        \item Resumo em língua vernácula (obrigatório);
        \item Resumo em língua estrangeira (obrigatório);
        \item Lista de ilustrações(opcional);
        \item Lista de tabelas (opcional);
        \item Lista de abreviaturas e siglas (opcional);
        \item Sumário (obrigatório).
    \end{enumerate}
\end{itemize}

\chapter{Mathematical Background}
\section{Math}
\blindmathtrue\blindtext[2]

\cref{tab:simple_table}
\begin{table}
    \centering
    \begin{tabular}{|c|c|c|}
        \hline
        \textbf{Column 1} & \textbf{Column 2} & \textbf{Column 3} \\
        \hline
        Row 1 & Row 1 & Row 1 \\
        Row 2 & Row 2 & Row 2 \\
        Row 3 & Row 3 & Row 3 \\
        \hline
    \end{tabular}
    \caption{A simple table}
    \label{tab:simple_table}
\end{table}


\cref{fig:simple_figure}
\begin{figure}
    \centering
    \includegraphics[width=0.5\textwidth]{template/logo.pdf}
    \caption{Macro logo}
    \label{fig:simple_figure}
\end{figure}

\subsection{Abstract Algebra}
\blindmathpaper
\subsubsection{Group Theory}
\blindtext[1]\blindmathfalse
\chapter{Development}
\Blindtext[2][1]
\chapter{Results}
\blindtext[2]

\chapter{Conclusions}
\blindtext[1]

\bibliographystyle{abbrvnat}
\bibliography{references}

\end{document}
