\documentclass[dissertation,master,firstlang=english,secondlang=brazil]{macrothesis}
% The package by default will load the glossaries and natbib packages
% If you want to disable them, use option `noglossaries` or `nonatbib`
\usepackage{blindtext}
\usepackage{amsmath,amsfonts,amssymb}
\usepackage{verbatim}
\usepackage{textcomp}
%%% Thesis metadata %%%
\title{A Minimal Thesis Example}
\author{John Doe}
\advisorname{Dr. Jane Smith}
\coadvisorname{Dr. Foo Bar} % If no co-advisor, just comment this line
% To add a second co-advisor, use the command as following:
% \coadvisorname[Second Co-advisor]{First Co-advisor }

% The following commands set the department, university, research group, program, and location names. The default values are exactly:
% \departmentname{Department of Electrical Engineering}
% \universityname{Universidade Federal de Minas Gerais}
% \researchgroupname{MACRO Research Group - Mechatronics, Control, and Robotics}
% \programname{Graduate Program in Electrical Engineering}
% \locationname{Belo Horizonte, Brazil}

%%% Preambles %%%
% Dedication and Acknowledgments
\dedication{To my family.}
\acknowledgments{
    This is an acknowledgments. If you are using another language as main 
    (not english), then you should use this command as following:

    ``\textbackslash acknowledgments[Acknowledgments in your language]\{Text\}''
}

% Abstracts (in both languages)
\abstract{
    This is the abstract in the first language. If you are using another language as main
    (not english), then you should use this command as following:

    ``\textbackslash abstract[Abstract in your language]\{Text\}''
}
\abstractsecond{
    Esse é um exemplo de abstract. Caso sua segunda língua não seja português,
    você deve usar o comando da seguinte forma:

    ``\textbackslash abstractsecond[Abstract in second language]\{Text\}''
}

%%% Acronyms and Notations %%%

% The package by default will load the glossaries package
% If you want to disable it, use option `noglossaries`
% To add new acronyms, use the command as following:
\newacronym{uav}{UAV}{Unmanned aerial vehicle}
\newacronym{dof}{DOF}{Degrees of freedom}
\newacronym{ufmg}{UFMG}{Universidade Federal de Minas Gerais}
% To add new notations, use the command as following:
\newnotation{Rn}{$\mathbb{R}^n$}{Euclidean space of dimension $n$}
\newnotation{frobnorm}{$\|\cdot\|_F$}{Frobenius norm}
\newnotation{N}{$\mathbb{N}$}{Set of natural numbers starting from zero or one depending on context}
\newnotation{Z}{$\mathbb{Z}$}{Set of all integers, both positive and negative including zero}
\newnotation{C}{$\mathbb{C}$}{Set of complex numbers, represented as $a + bi$ where $a, b \in \mathbb{R}$}
% You do not need to use the \gls{label} command, as the package will 
% automatically add the acronyms and notations to the list, whether they are
% used in the text or not.
% NOTE: If you use glossaries, every new entry must be defined before the
% \begin{document} command, i.e., here in the preamble.

% If you disable glossaries, the package will assume that there is a file
% Acronyms.tex and Notations.tex in the same folder as the main file
% We assume by default the titles Acronyms and Notations



\begin{document}
% Generate cover page
\maketitlepage
% Generate preamble page (dedication, acknowledgments, abstracts, 
% list of figures, list of tables, acronyms, and notations)
\preamblepage

%----------------------------------------------------------------------------------------
%	THESIS CONTENT - CHAPTERS
%----------------------------------------------------------------------------------------
\chapter{Introduction}
Para utilizar o pacote, simplesmente adicione o comando \texttt{\textbackslash usepackage\{macrothesis\}} ao preâmbulo do seu documento. Não delete o arquivo \texttt{template/logo.pdf}, já que ele é utilizado para gerar a capa do documento. 
\section{Opções do pacote}
O pacote \texttt{macrothesis} possui algumas opções que podem ser utilizadas para personalizar o documento. As opções disponíveis são:
\begin{itemize}
    \item \texttt{noglossaries}: Desabilita o pacote \texttt{glossaries} que é utilizado para gerenciar acrônimos e notações. Esta opção é útil caso você deseje utilizar um pacote diferente para gerenciar acrônimos e notações ou caso você deseje utilizar outras opções para o pacote. É possível definir uma página de acrônimos e notações manualmente utilizando os comandos \texttt{\textbackslash setnotationpage}.
    \item \texttt{nonatbib}: Desabilita o pacote \texttt{natbib} que é utilizado para gerenciar referências bibliográficas. Esta opção é útil caso você deseje utilizar um pacote diferente para gerenciar referências bibliográficas ou caso você deseje utilizar outras opções para o pacote.
    \item \texttt{nohyperref}: Desabilita o pacote \texttt{hyperref} que é utilizado para gerar links no documento. Esta opção é útil caso você deseje utilizar um pacote diferente para gerar links ou caso você deseje utilizar outras opções para o pacote.
    \item \texttt{dissertation}: Usa o termo ``Dissertation'' ao invés do padrão ``Thesis''.
    \item \texttt{thesis}: Usa o termo ``Thesis''. Não é necessário utilizar essa opção, pois é o padrão.
    \item \texttt{doctor}: Usa o título ``Doctor'' ao invés do padrão ``Master''.
    \item \texttt{master}: Usa o título ``Master''. Não é necessário utilizar essa opção, pois é o padrão.
    \item \texttt{firstlang=<lingua>}: Define a língua principal do documento. Por padrão, a língua principal é o inglês. As opções disponíveis são: \\\emph{\supportedlangs}. 
    \item \texttt{secondlang=<lingua>}: Define a língua secundária do documento. As opções disponíveis são as mesmas. Por padrão, a língua secundária é \texttt{none}, o que significa que não há uma língua secundária.
\end{itemize}
Note que as opções de línguas somente automatizam os títulos como ``Abstract'', ``Resumo'', ``Contents'', etc. É possível escrever o texto em outras línguas, desde que esses títulos sejam alterados manualmente. Alguns ambientes já possuem opções para alterar o título manualmente.

\subsection{Uso do pacote \texttt{glossaries}}
Caso deseje utilizar o pacote \texttt{glossaries}, você pode adicionar novos acrônimos e notações utilizando os comandos predefinidos \texttt{\textbackslash newacronym\{label\}\{sigla\}\{descrição\}} e \texttt{\textbackslash newnotation\{label\}\{notação\}\{descrição\}}, respectivamente. Por exemplo, o acrônimo \gls{uav} e a notação \gls{Rn} foram adicionados utilizando esses comandos. Os acrônimos serão ordenados em ordem alfabética e as notações em ordem de definição.

Não é necessário utilizar o comando \texttt{\textbackslash gls\{label\}} (ou algo similar) para referenciar os acrônimos e notações, pois o pacote \texttt{macrothesis} irá automaticamente adicionar todos os acrônimos e notações à lista, independentemente de serem utilizados no texto ou não. Apesar disso, você \textbf{deve} definir um label único para cada acrônimo e notação.

Esses termos devem ser adicionados no preâmbulo (antes do \verb|begin{document}|) da seguinte forma:
\begin{verbatim}
\newacronym{uav}{UAV}{Unmanned aerial vehicle}
\newacronym{dof}{DOF}{Degrees of freedom}
\newacronym{ufmg}{UFMG}{Universidade Federal de Minas Gerais}

\newnotation{Rn}{$\mathbb{R}^n$}{Euclidean space of dimension $n$}
\newnotation{frobnorm}{$\|\cdot\|_F$}{Frobenius norm}

\begin{document}
...
\end{document}
\end{verbatim}
É possível adicionar as notações e acrônimos em arquivos separados, importando-se esses arquivos antes do preâmbulo utilizando \texttt{}

Note que o pacote \texttt{glossaries} será carregado por padrão com as opções \texttt{acronym} e \texttt{symbols}. Caso deseje modificar essas opções, você deve utilizar a opção \texttt{noglossaries} do pacote \texttt{macrothesis} e carregar o pacote \texttt{glossaries} manualmente.

\subsection{Uso do pacote \texttt{natbib}}
Caso deseje utilizar o pacote \texttt{natbib}, você pode utilizar os comandos \texttt{\textbackslash citep\{label\}} para citações indiretas
e \texttt{\textbackslash citet\{label\}} para citações diretas. Por exemplo, a referência \citep{Gallier2020} foi adicionada utilizando o comando \texttt{\textbackslash citep\{test\}} e a referência \citet{Lee2012} foi adicionada utilizando o comando \texttt{\textbackslash citet\{test\}}.

O pacote \texttt{natbib} é carregado por padrão com as opções \texttt{authoryear} e \texttt{round}. Caso deseje modificar essas opções, você deve utilizar a opção \texttt{nonatbib} do pacote \texttt{macrothesis} e carregar o pacote \texttt{natbib} manualmente.

\section{Uso do pacote \texttt{macrothesis}}
O pacote \texttt{macrothesis} é utilizado para gerar a capa, a folha de rosto, a dedicatória, os agradecimentos, os resumos, a lista de figuras, a lista de tabelas, a lista de acrônimos, a lista de notações, e a bibliografia. Para utilizar o pacote, basta adicionar os comandos \texttt{\textbackslash maketitlepage} e \texttt{\textbackslash preamblepage} ao início do seu documento (nessa ordem).

Para alterar o título, o autor, o orientador, etc. do documento, você deve utilizar os comandos \texttt{\textbackslash title\{Título\}}, \texttt{\textbackslash author\{Autor\}}, \texttt{\textbackslash advisorname\{Orientador\}}, etc. no preâmbulo. Por exemplo, para gerar a capa e a folha de rosto, você deve utilizar os comandos:
\begin{verbatim}
    \title{A Minimal Thesis Example}
    \author{John Doe}
    \advisorname{Dr. Jane Smith}
    \coadvisorname{Dr. Foo Bar}
    \universityname{Universidade Federal de Minas Gerais}
    \researchgroupname{MACRO Research Group - Mechatronics,
     Control, and Robotics}
    \programname{Graduate Program in Electrical Engineering}
    \locationname{Belo Horizonte, Brazil}
    ...
    \begin{document}
    \maketitlepage
    \preamblepage
    ...
    \end{document}
\end{verbatim}
O comando \texttt{\textbackslash coadvisorname\{Coorientador\}} é opcional e deve ser utilizado caso haja um coorientador. Para adicionar um segundo coorientador, você deve utilizar o comando da seguinte forma: \texttt{\textbackslash coadvisorname[Segundo Coorientador]\{Primeiro Coorientador\}}.

Para adicionar a dedicatória, os agradecimentos, e os resumos, você deve utilizar os comandos:
\begin{itemize}
    \item \texttt{\textbackslash dedication\{Dedicatória\}}
    \item \texttt{\textbackslash acknowledgments\{Agradecimentos\}}
    \item \texttt{\textbackslash abstract\{Resumo\}}
    \item \texttt{\textbackslash abstractsecond\{Abstract\}}
\end{itemize}

Por exemplo, para adicionar a dedicatória e os agradecimentos, você deve utilizar os comandos:
\begin{verbatim}
    \dedication{To my family.}
    \acknowledgments{
        Long and emotional text here...
    }
    \abstract{    
        Abstract in english...
    }
    \abstractsecond{
        Resumo em português...
    }
    ...
    \begin{document}
    \maketitlepage
    \preamblepage
    ...
    \end{document}
\end{verbatim}
Caso necessário, é possível modificar o título das seções de abstracts e acknowledgments utilizando os comandos da seguinte forma: \texttt{\textbackslash abstract[Título em sua língua]\{Texto\}} e \texttt{\textbackslash acknowledgments[Título em sua língua]\{Texto\}}.

\section{TODO}
O pacote ainda está em desenvolvimento e algumas funcionalidades ainda não foram implementadas. Algumas das funcionalidades que ainda serão implementadas são:
\begin{itemize}
    % \item Adicionar opções para alterar as línguas utilizadas no documento.
    % \item Adicionar opção para selecionar "Thesis" ou "Dissertation". Além de selecionar "Master" ou "Doctor".
    % \item Melhorar a forma como a opção \texttt{noglossaries} é tratada.
    % \item Adicionar opções para o pacote \texttt{hyperref} ou adicionar uma opção para desabilitar o pacote.
    \item Traduzir automaticamente Dissertation/Thesis, Master/Doctor e o texto da capa;
    \item Somente mostrar a lista de figuras e a lista de tabelas caso haja figuras e tabelas no documento.
    \item Melhorar o visual das listas de acrônimos e notações.
\end{itemize}

\chapter{Mathematical Background}
\section{Math}
\blindmathtrue\blindtext[2]
\subsection{Abstract Algebra}
\blindmathpaper
\subsubsection{Group Theory}
\blindtext[1]\blindmathfalse
\chapter{Development}
\Blindtext[2][1]
\chapter{Results}
\blindtext[2]

\chapter{Conclusions}
\blindtext[1]

\bibliographystyle{plainnat}
\bibliography{references}

\end{document}
